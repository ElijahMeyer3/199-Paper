% Options for packages loaded elsewhere
\PassOptionsToPackage{unicode}{hyperref}
\PassOptionsToPackage{hyphens}{url}
\PassOptionsToPackage{dvipsnames,svgnames,x11names}{xcolor}
%
\documentclass[
  12pt]{article}

\usepackage{amsmath,amssymb}
\usepackage{lmodern}
\usepackage{iftex}
\ifPDFTeX
  \usepackage[T1]{fontenc}
  \usepackage[utf8]{inputenc}
  \usepackage{textcomp} % provide euro and other symbols
\else % if luatex or xetex
  \usepackage{unicode-math}
  \defaultfontfeatures{Scale=MatchLowercase}
  \defaultfontfeatures[\rmfamily]{Ligatures=TeX,Scale=1}
\fi
% Use upquote if available, for straight quotes in verbatim environments
\IfFileExists{upquote.sty}{\usepackage{upquote}}{}
\IfFileExists{microtype.sty}{% use microtype if available
  \usepackage[]{microtype}
  \UseMicrotypeSet[protrusion]{basicmath} % disable protrusion for tt fonts
}{}
\makeatletter
\@ifundefined{KOMAClassName}{% if non-KOMA class
  \IfFileExists{parskip.sty}{%
    \usepackage{parskip}
  }{% else
    \setlength{\parindent}{0pt}
    \setlength{\parskip}{6pt plus 2pt minus 1pt}}
}{% if KOMA class
  \KOMAoptions{parskip=half}}
\makeatother
\usepackage{xcolor}
\setlength{\emergencystretch}{3em} % prevent overfull lines
\setcounter{secnumdepth}{5}
% Make \paragraph and \subparagraph free-standing
\ifx\paragraph\undefined\else
  \let\oldparagraph\paragraph
  \renewcommand{\paragraph}[1]{\oldparagraph{#1}\mbox{}}
\fi
\ifx\subparagraph\undefined\else
  \let\oldsubparagraph\subparagraph
  \renewcommand{\subparagraph}[1]{\oldsubparagraph{#1}\mbox{}}
\fi


\providecommand{\tightlist}{%
  \setlength{\itemsep}{0pt}\setlength{\parskip}{0pt}}\usepackage{longtable,booktabs,array}
\usepackage{calc} % for calculating minipage widths
% Correct order of tables after \paragraph or \subparagraph
\usepackage{etoolbox}
\makeatletter
\patchcmd\longtable{\par}{\if@noskipsec\mbox{}\fi\par}{}{}
\makeatother
% Allow footnotes in longtable head/foot
\IfFileExists{footnotehyper.sty}{\usepackage{footnotehyper}}{\usepackage{footnote}}
\makesavenoteenv{longtable}
\usepackage{graphicx}
\makeatletter
\def\maxwidth{\ifdim\Gin@nat@width>\linewidth\linewidth\else\Gin@nat@width\fi}
\def\maxheight{\ifdim\Gin@nat@height>\textheight\textheight\else\Gin@nat@height\fi}
\makeatother
% Scale images if necessary, so that they will not overflow the page
% margins by default, and it is still possible to overwrite the defaults
% using explicit options in \includegraphics[width, height, ...]{}
\setkeys{Gin}{width=\maxwidth,height=\maxheight,keepaspectratio}
% Set default figure placement to htbp
\makeatletter
\def\fps@figure{htbp}
\makeatother

\addtolength{\oddsidemargin}{-.5in}%
\addtolength{\evensidemargin}{-1in}%
\addtolength{\textwidth}{1in}%
\addtolength{\textheight}{1.7in}%
\addtolength{\topmargin}{-1in}%
\makeatletter
\makeatother
\makeatletter
\makeatother
\makeatletter
\@ifpackageloaded{caption}{}{\usepackage{caption}}
\AtBeginDocument{%
\ifdefined\contentsname
  \renewcommand*\contentsname{Table of contents}
\else
  \newcommand\contentsname{Table of contents}
\fi
\ifdefined\listfigurename
  \renewcommand*\listfigurename{List of Figures}
\else
  \newcommand\listfigurename{List of Figures}
\fi
\ifdefined\listtablename
  \renewcommand*\listtablename{List of Tables}
\else
  \newcommand\listtablename{List of Tables}
\fi
\ifdefined\figurename
  \renewcommand*\figurename{Figure}
\else
  \newcommand\figurename{Figure}
\fi
\ifdefined\tablename
  \renewcommand*\tablename{Table}
\else
  \newcommand\tablename{Table}
\fi
}
\@ifpackageloaded{float}{}{\usepackage{float}}
\floatstyle{ruled}
\@ifundefined{c@chapter}{\newfloat{codelisting}{h}{lop}}{\newfloat{codelisting}{h}{lop}[chapter]}
\floatname{codelisting}{Listing}
\newcommand*\listoflistings{\listof{codelisting}{List of Listings}}
\makeatother
\makeatletter
\@ifpackageloaded{caption}{}{\usepackage{caption}}
\@ifpackageloaded{subcaption}{}{\usepackage{subcaption}}
\makeatother
\makeatletter
\@ifpackageloaded{tcolorbox}{}{\usepackage[many]{tcolorbox}}
\makeatother
\makeatletter
\@ifundefined{shadecolor}{\definecolor{shadecolor}{rgb}{.97, .97, .97}}
\makeatother
\makeatletter
\makeatother
\ifLuaTeX
  \usepackage{selnolig}  % disable illegal ligatures
\fi
\usepackage[]{natbib}
\bibliographystyle{agsm}
\IfFileExists{bookmark.sty}{\usepackage{bookmark}}{\usepackage{hyperref}}
\IfFileExists{xurl.sty}{\usepackage{xurl}}{} % add URL line breaks if available
\urlstyle{same} % disable monospaced font for URLs
\hypersetup{
  pdftitle={Audience},
  pdfauthor={Elijah Meyer; Mine Çetinkaya-Rundel},
  pdfkeywords={Data Science, Curriculum, Pedagogy},
  colorlinks=true,
  linkcolor={blue},
  filecolor={Maroon},
  citecolor={Blue},
  urlcolor={Blue},
  pdfcreator={LaTeX via pandoc}}


\begin{document}


\def\spacingset#1{\renewcommand{\baselinestretch}%
{#1}\small\normalsize} \spacingset{1}


%%%%%%%%%%%%%%%%%%%%%%%%%%%%%%%%%%%%%%%%%%%%%%%%%%%%%%%%%%%%%%%%%%%%%%%%%%%%%%

\date{February 21, 2023}
\title{\bf Audience}
\author{
Elijah Meyer\\
Department of Statistics, Duke University\\
and\\Mine Çetinkaya-Rundel\\
Department of Statistics, Duke University\\
}
\maketitle

\bigskip
\bigskip
\begin{abstract}
The text of your abstract. 200 or fewer words.
\end{abstract}

\noindent%
{\it Keywords:} Data Science, Curriculum, Pedagogy
\vfill

\newpage
\spacingset{1.9} % DON'T change the spacing!
\ifdefined\Shaded\renewenvironment{Shaded}{\begin{tcolorbox}[breakable, frame hidden, boxrule=0pt, borderline west={3pt}{0pt}{shadecolor}, sharp corners, interior hidden, enhanced]}{\end{tcolorbox}}\fi

Audience - new to teaching / teachers increasing course size (mention
this as a draw in because this is happening). This could be a hook.

I think this needs to be further brought out in the beginning of the
intro when formalling writing\ldots.

\hypertarget{sec-intro}{%
\section{Introduction}\label{sec-intro}}

Reader should want to continue reading by thinking ``this is something
that's interesting''

-- \textbf{Why} we are writing this paper

There is a demand for data scientists. An estimated 11.5 million new
data science jobs are projected to be created by 2026, while employment
of data scientists is projected to grow by 36 percent from 2021 to 2031
\citep{labor_2022}. \textbf{As demand increases, class sizes to best
prepare students for these positions increases as well}. The increasing
volume of enrollment of data science students \citep{Redmond2022}
requires that statistics and data science educators commit to developing
modern curriculum in order to help students be successful. Despite the
demand, colleges are still struggling with what a modern data science
curriculum should look like \citep{Schwab2020}, and how it can be
effectively taught to a large student audience. To this point, much more
thought, work, and discussions need to take place there before a
consensus is reached on what should a modern data science curricula
should be.

\textbf{What people have been investigating / what's recommended}

Curricular

\begin{itemize}
\item
  College of Charleston, South Carolina \citep{Anderson2014}
\item
  Intro to Data Science implementation \citep{Asamoah2015}
\item
  Curriculum Guidelines for Undergraduate Programs in Data Science
  \citep{veaux_2017}
\item
  50 Years of Data Science \citep{Donoho_2017}
\item
  Association for Computing Machinery (ACM) Education Council
  \citep{Danyluk_2021}
\end{itemize}

Pedagogical

\begin{itemize}
\item
  Computational and Tech recommendations in curricula
  \citep{Donoho_2017, Beckman_2021}
\item
  Active Learning / Other recommendations
  \citep{Dogucu_2022, CetinkayaRundel_2022} + GAISE (Carver, 2016);
  Conference Board of Mathematical Sciences, 2016).
\end{itemize}

\textbf{But\ldots{}}

However, the majority of context across the landscape of data science
curricula largely focuses on how to model data \citep{Donoho2017}. This,
in combination when a lack of consensus on what constitutes a data
science curricula, and even less research on how to develop modern data
science course presents a need for a blueprint to design and implement
and modernized introduction to data science course.

\textbf{So}

The need for those who can make meaning of data are clear. Thus, it is
critical that data science education provides programs that adequately
prepare and train students in the field of data science. This article
adds to the existing literature by describing a modernized introductory
data science course and how its taught, at Duke University.

The purpose of this paper is to offer valuable structure, while
providing experiences from our perspective. It is of importance that we
stress the amount of flexibility and strength in individuality in all
aspects when creating, designing, and implementing an introductory data
science course. (I don't know where this goes yet; Mine brings up a good
point to make sure this is somewhere highlighted at the forefront)

In this paper, we discuss the integration of technology in our
Introductory to Data Science course, and how these choices have helped
shape our curricular and pedagogical decisions made. This includes
detailing the implementation of the Kaplan Way learning model, to
support a large class of students with a diverse background in
statistics, data science, and coding experience. Within this format, we
provide examples of and describe activities and assessments given both
in and outside of class. We extend discussions and provide
recommendations for implementing and integrating computing tools, such
as R-studio and GitHub, through our experiences in our course. Lastly,
we discuss challenges, and provide insight to help faculty wanting to
adopt or adapt a course similar to introductory to data science at Duke
University. The purpose of this paper is to continue the discussion, and
present a modernized curricula for an introductory data science course
at Duke University, and the pedagogical decisions to help best equip
students with the data science skills necessary for future classes.

\hypertarget{sec-course}{%
\section{The Course}\label{sec-course}}

Reader should want to continue reading by thinking ``this is something
that I would like to teach''

\textbf{Hook} Students enroll in a large class capacity with little to
no statistics, data science, or coding experience. These factors are two
common hurdles identified by faculty when trying to implement a data
science course \citep{Schwab2020, Kok_2008}. By the end of this course,
students are able to use R, R-studio, and GitHub to clean, investigate,
and communicate with data in a reproducible manner while answering a
targeted research question.

-- Detailed learning objectives of this course include learning to
explore, visualize, and analyze data in a reproducible and shareable
manner through the use of R-studio and GitHub \citep{R21, github}.

-- gain experience in data wrangling, exploratory data analysis,
predictive modeling, and data visualization

-- The Kaplan Way is a learning model ``that combines a scientific,
evidence-based design philosophy with a straightforward educational
approach to learning'' \citep{schweser_2023}

-- Dedicated active learning approach

-- have a section on the content we teach and cite a paper (maybe goes
here)

In the following sections, we first detail the teaching team used to
instruct Introduction to Data Science. Next, we detail the technology
we've chosen to use when creating and facilitating our introductory data
science course. Then, we discuss our pedagogical choices that go into a
typical week of our Introduction to Data Science Course. Finally, we
discuss the preparation work needed to teach our Introduction to Data
Science, informing those on how to get started when creating and
modifying their own introductory data science course

\hypertarget{teaching-team}{%
\section{Teaching Team}\label{teaching-team}}

-- What is needed

-- What are their responsibilities

-- Instructor; Head-TAs; Course Organizers; Lab Leaders; Lab Helpers

\begin{itemize}
\item
  Preparation: How to
\item
  Communication: How to + Recommendations / strategies
\end{itemize}

\hypertarget{sec-tech}{%
\section{Technology}\label{sec-tech}}

Reader should want to continue reading by thinking ``so how do I use
this''

Next, we discuss our choices of technology\ldots..

\begin{itemize}
\tightlist
\item
  Github: Why version control; what it is; how we use it at the
  beginnings of the course
\end{itemize}

-- Collecting student information

-- Setting up a GitHub organization

\begin{itemize}
\tightlist
\item
  R \& R-studio; Why; what is is; how students get set up with it
\end{itemize}

-- Creating Duke Containers

(Mention alternative R-studio Cloud in discussion?)

Each of these types of technology aids in the creation and
implementation our data science pedagogy. This includes in-class
application exercise (AEs), lectures, labs, and assessments within the
Kaplan model of learning.

\textbf{INSERT IMAGE HERE}

Unpack the image below

\hypertarget{sec-ped}{%
\section{Pedagogy}\label{sec-ped}}

-- Reader should want to continue reading by thinking ``so how do I
teach this''

In this course, we have chosen a combination of teaching methods,
interactive activities, and learning assessments to help best prepare
introductory data science students the tools they need to be successful
outside of university or in future coursework. Our pedagogy includes
facilitating in-class AEs, facilitating lectures, running a Lab, and
assigning assessments to provide students an opportunity to show what
they've learned.

\hypertarget{application-exercies-aes}{%
\subsection{Application Exercies (AEs)}\label{application-exercies-aes}}

A majority of the time in class will be dedicated to working on AEs.
These exercises are live-coding opportunities to practice applying data
science concepts and code introduced through preparation materials. AEs
are no-stakes assessments, often graded on completion, where students
have the ability to write and edit code while asking questions at the
student or class level and receive immediate feedback.

-- How GitHub and R are used to create AEs for students

It is up to the discretion of the instructor on the content that goes
into the AEs. This can include having students write code themselves,
fill in the blank coding exercises, commenting on complete code, or a
combination of such questions.

-- Kaplan Way (Practice) / Active learning implementation of AEs
(typical class day strategies / discussion)

\hypertarget{lecture}{%
\subsection{Lecture}\label{lecture}}

A typical week yields 75-minute lectures on Mondays and Wednesdays.
These lectures take up part of the class time, and are designed to
introduce new concepts or review topics from the preparation videos in a
more traditional format.

-- Creating slides using Quarto in R (?)

-- Kaplan Way (Prepare) / Active learning implementation during lecture
(typical class day strategies / discussion)

Lectures are recorded and made available to students with an excused
absence upon request.

\hypertarget{labs}{%
\subsection{Labs}\label{labs}}

-- apply the concepts discussed in lecture to various data analysis
scenarios, with a focus on the computation

-- Individual and team based

-- Repo creation

-- Kaplan Way (Perform) (typical class day strategies / discussion)

\hypertarget{assessments}{%
\subsection{Assessments}\label{assessments}}

-- Kaplan Way (Perform)

-- HW

\begin{itemize}
\tightlist
\item
  apply what you've learned during lecture and lab to complete data
  analysis tasks
\end{itemize}

-- Lab

\begin{itemize}
\item
  apply the concepts discussed in lecture to various data analysis
  scenarios, with a focus on the computation
\item
  Individual and team based
\item
  Repo creation
\end{itemize}

-- Exams

\begin{itemize}
\item
  opportunity to demonstrate what you've learned in the course thus far
\item
  take home exams
\end{itemize}

-- Project

\begin{itemize}
\item
  analyze an interesting data-driven research question
\item
  group project
\item
  describe the process / expectations of the project
\end{itemize}

\^{}\^{} The writing above should be in a form where there is practical
/ useful information for the reader. If they were designing the course,
do they better understand the assessment structure of an Intro to Data
Science course?

\hypertarget{discussion}{%
\section{Discussion}\label{discussion}}

-- Pros (What works well)

Ex. Live coding good

-- Cons (What could be improved)

Ex. Students can fall behind during live coding sessions

-- Experiences

Examples include

-- Feasible to adopt to teach for multiple semesters as data science
continues to evolve. There is value in the investment.

-- Computational Resources

-- Human Resources

\newpage


  \bibliography{bibliography.bib}


\end{document}
