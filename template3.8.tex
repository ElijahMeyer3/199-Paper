% Options for packages loaded elsewhere
\PassOptionsToPackage{unicode}{hyperref}
\PassOptionsToPackage{hyphens}{url}
\PassOptionsToPackage{dvipsnames,svgnames,x11names}{xcolor}
%
\documentclass[
  12pt]{article}

\usepackage{amsmath,amssymb}
\usepackage{lmodern}
\usepackage{iftex}
\ifPDFTeX
  \usepackage[T1]{fontenc}
  \usepackage[utf8]{inputenc}
  \usepackage{textcomp} % provide euro and other symbols
\else % if luatex or xetex
  \usepackage{unicode-math}
  \defaultfontfeatures{Scale=MatchLowercase}
  \defaultfontfeatures[\rmfamily]{Ligatures=TeX,Scale=1}
\fi
% Use upquote if available, for straight quotes in verbatim environments
\IfFileExists{upquote.sty}{\usepackage{upquote}}{}
\IfFileExists{microtype.sty}{% use microtype if available
  \usepackage[]{microtype}
  \UseMicrotypeSet[protrusion]{basicmath} % disable protrusion for tt fonts
}{}
\makeatletter
\@ifundefined{KOMAClassName}{% if non-KOMA class
  \IfFileExists{parskip.sty}{%
    \usepackage{parskip}
  }{% else
    \setlength{\parindent}{0pt}
    \setlength{\parskip}{6pt plus 2pt minus 1pt}}
}{% if KOMA class
  \KOMAoptions{parskip=half}}
\makeatother
\usepackage{xcolor}
\setlength{\emergencystretch}{3em} % prevent overfull lines
\setcounter{secnumdepth}{5}
% Make \paragraph and \subparagraph free-standing
\ifx\paragraph\undefined\else
  \let\oldparagraph\paragraph
  \renewcommand{\paragraph}[1]{\oldparagraph{#1}\mbox{}}
\fi
\ifx\subparagraph\undefined\else
  \let\oldsubparagraph\subparagraph
  \renewcommand{\subparagraph}[1]{\oldsubparagraph{#1}\mbox{}}
\fi


\providecommand{\tightlist}{%
  \setlength{\itemsep}{0pt}\setlength{\parskip}{0pt}}\usepackage{longtable,booktabs,array}
\usepackage{calc} % for calculating minipage widths
% Correct order of tables after \paragraph or \subparagraph
\usepackage{etoolbox}
\makeatletter
\patchcmd\longtable{\par}{\if@noskipsec\mbox{}\fi\par}{}{}
\makeatother
% Allow footnotes in longtable head/foot
\IfFileExists{footnotehyper.sty}{\usepackage{footnotehyper}}{\usepackage{footnote}}
\makesavenoteenv{longtable}
\usepackage{graphicx}
\makeatletter
\def\maxwidth{\ifdim\Gin@nat@width>\linewidth\linewidth\else\Gin@nat@width\fi}
\def\maxheight{\ifdim\Gin@nat@height>\textheight\textheight\else\Gin@nat@height\fi}
\makeatother
% Scale images if necessary, so that they will not overflow the page
% margins by default, and it is still possible to overwrite the defaults
% using explicit options in \includegraphics[width, height, ...]{}
\setkeys{Gin}{width=\maxwidth,height=\maxheight,keepaspectratio}
% Set default figure placement to htbp
\makeatletter
\def\fps@figure{htbp}
\makeatother

\addtolength{\oddsidemargin}{-.5in}%
\addtolength{\evensidemargin}{-1in}%
\addtolength{\textwidth}{1in}%
\addtolength{\textheight}{1.7in}%
\addtolength{\topmargin}{-1in}%
\makeatletter
\makeatother
\makeatletter
\makeatother
\makeatletter
\@ifpackageloaded{caption}{}{\usepackage{caption}}
\AtBeginDocument{%
\ifdefined\contentsname
  \renewcommand*\contentsname{Table of contents}
\else
  \newcommand\contentsname{Table of contents}
\fi
\ifdefined\listfigurename
  \renewcommand*\listfigurename{List of Figures}
\else
  \newcommand\listfigurename{List of Figures}
\fi
\ifdefined\listtablename
  \renewcommand*\listtablename{List of Tables}
\else
  \newcommand\listtablename{List of Tables}
\fi
\ifdefined\figurename
  \renewcommand*\figurename{Figure}
\else
  \newcommand\figurename{Figure}
\fi
\ifdefined\tablename
  \renewcommand*\tablename{Table}
\else
  \newcommand\tablename{Table}
\fi
}
\@ifpackageloaded{float}{}{\usepackage{float}}
\floatstyle{ruled}
\@ifundefined{c@chapter}{\newfloat{codelisting}{h}{lop}}{\newfloat{codelisting}{h}{lop}[chapter]}
\floatname{codelisting}{Listing}
\newcommand*\listoflistings{\listof{codelisting}{List of Listings}}
\makeatother
\makeatletter
\@ifpackageloaded{caption}{}{\usepackage{caption}}
\@ifpackageloaded{subcaption}{}{\usepackage{subcaption}}
\makeatother
\makeatletter
\@ifpackageloaded{tcolorbox}{}{\usepackage[many]{tcolorbox}}
\makeatother
\makeatletter
\@ifundefined{shadecolor}{\definecolor{shadecolor}{rgb}{.97, .97, .97}}
\makeatother
\makeatletter
\makeatother
\ifLuaTeX
  \usepackage{selnolig}  % disable illegal ligatures
\fi
\usepackage[]{natbib}
\bibliographystyle{agsm}
\IfFileExists{bookmark.sty}{\usepackage{bookmark}}{\usepackage{hyperref}}
\IfFileExists{xurl.sty}{\usepackage{xurl}}{} % add URL line breaks if available
\urlstyle{same} % disable monospaced font for URLs
\hypersetup{
  pdftitle={Introductory Data Science: A Blueprint to Navigate Curricular, Pedagogical, and Computational Challenges},
  pdfauthor={Elijah Meyer; Mine Çetinkaya-Rundel},
  pdfkeywords={Data Science, Curriculum, Pedagogy},
  colorlinks=true,
  linkcolor={blue},
  filecolor={Maroon},
  citecolor={Blue},
  urlcolor={Blue},
  pdfcreator={LaTeX via pandoc}}


\begin{document}


\def\spacingset#1{\renewcommand{\baselinestretch}%
{#1}\small\normalsize} \spacingset{1}


%%%%%%%%%%%%%%%%%%%%%%%%%%%%%%%%%%%%%%%%%%%%%%%%%%%%%%%%%%%%%%%%%%%%%%%%%%%%%%

\date{March 8, 2023}
\title{\bf Introductory Data Science: A Blueprint to Navigate
Curricular, Pedagogical, and Computational Challenges}
\author{
Elijah Meyer\\
Department of Statistics, Duke University\\
and\\Mine Çetinkaya-Rundel\\
Department of Statistics, Duke University\\
}
\maketitle

\bigskip
\bigskip
\begin{abstract}
The text of your abstract. 200 or fewer words.
\end{abstract}

\noindent%
{\it Keywords:} Data Science, Curriculum, Pedagogy
\vfill

\newpage
\spacingset{1.9} % DON'T change the spacing!
\ifdefined\Shaded\renewenvironment{Shaded}{\begin{tcolorbox}[enhanced, breakable, frame hidden, boxrule=0pt, sharp corners, borderline west={3pt}{0pt}{shadecolor}, interior hidden]}{\end{tcolorbox}}\fi

\hypertarget{sec-course}{%
\section{The Course}\label{sec-course}}

Reader should want to continue reading by thinking ``this is something
that I would like to teach''

In the following sections, we describe our introductory data science
course offered to predominantly freshman level students at Duke
University called Introduction to Data Science and Statistical Thinking
(STA199). Often, these students are (interested in / major + minor in /
description of typical student demographic). Students enrolling in
STA199 commonly have little to no statistics, data science, or coding
experience. In a typical semester, this course seats roughly 179
students, which is considered to be large by all measures. Both the lack
of experience and size of class are identified as two common hurdles by
faculty when trying to implement an introductory data science course
\citep{Schwab2020, Kok_2008}. However, by the end of this course,
students are able to use data both R and GitHub to understand natural
phenomena, investigate patterns, and model outcomes in a reproducible
format. \textbackslash{}

This course is built on four large-scale learning objectives: learn to
explore, visualize, and analyze data in a reproducible and shareable
manner; gain experience in data wrangling and munging, exploratory data
analysis, predictive modeling, and data visualization; work on problems
and case studies inspired by and based on real-world questions and data;
learn to effectively communicate results through written assignments and
project presentation. These objectives are accomplished through
interactive lectures and labs that present content, problems, and case
studies inspired by and based on real-world questions and data.

When teaching, instructors are committed to the Kaplan Way learning
model ``that combines a scientific, evidence-based design philosophy
with a straightforward educational approach to learning'' \citep[pg.
3]{schweser_2023}.

\begin{figure}

{\centering \includegraphics{images/Kaplan.png}

}

\end{figure}

This model is composed of three phases: prepare, practice, and perform.
All three phases are designed to guide the instructor in facilitating an
overall quality learning experience for the student. Each of these
phases are performed during a typical week within a semester, including
through preparation material, lectures, in-class application exercises,
labs, and assessments such as homework.

During the prepare phases, students are completing readings, watching
videos, and listening to lecture that ultimately builds upon a new
foundation of data science concepts being learned. The goal of preparing
students is to put them in a position where they can build upon what
they are learning, and create new knowledge through the practice phase.
The practice phase is designed to be an opportunity for students to
reinforce the new information gained, as well as uncover new concepts in
data science. This is achieved through the use of interactive
application exercises (AEs) in class where students work alone, in
groups, or with the instructor in live coding sessions. Finally,
students enter the perform phase to show their progress made in the
previous two phases. This is typically done through assessments such as
homework, exams, and projects. Additionally, students perform
individually, or with a group, during weekly labs that tend to focus
more on computation. This learning model is a continuous cycle
throughout the semester as new topics are introduced.

Topics taught in STA199 fall under two major units: Unit 1 - Exploring
data; Unit 2 - Making rigorous conclusions. In Unit 1, students become
first introduced to R, R-studio, and Github. During this unit, students
start to create data visualizations and learn how to both import and
manipulate data to be better suited for analysis. In unit 2, students
extend their investigations with data to include modeling. Specifically,
students fit a variety of models (simple linear regression, multiple
linear regression, logistic regression), and learn the fundamentals of
hypothesis testing. For a more complete description about the topics
taught and data sets used in creating these lessons, please see
\textbf{A fresh look at introductory data science (cite)}.

In this paper, we describe the preparation and implementation process of
STA199 in its entirety. This includes details of the teaching team used
to instruct STA199, technology chosen to use when creating and
facilitating our introductory data science course, and the pedagogical
choices that go into a typical week of teaching. This includes a
comprehensive description of a flexible framework on how to create, set
up, and implement an introduction to data science course similar to
STA199. When describing this framework, we articulate first hand
experiences and suggestions surrounding some of the choices made to
create and instruct STA199.

\hypertarget{teaching-team}{%
\section{Teaching Team}\label{teaching-team}}

We define a teaching team as a group consisting of one instructor and
multiple teaching assistants (TAs). The assignment of any teaching
assistant is to both support the instructor in charge of the class, and
support the students in the classroom. These TAs range from
undergraduate to PH.D. level students, and vary in teaching experience.
(Writing on TA selection process). Once selected\ldots{} (writing on TA
training).

Once training is complete, students are assigned roles that indicate
their responsibilities during the semester. These roles include
\emph{course organizer}, \emph{head TA}, \emph{lab leader}, and
\emph{lab helper}. Often, these roles are given based on the level of
student, with more academically experienced TAs taking on the roles of
course organizer, head TA, and lab leader, where as students with less
experience (i.e.~undergraduate students) take on the role of lab helper.

Lab sections are held once a week, and are facilitated in person by both
a lab leader and lab helper. The responsibilities of a lab helper are
supporting both the students and lab leader as they see fit. Examples of
this may include setting up the classroom before class, or conducting
small group conversations when students have questions about the
material. The lab leader is responsible for facilitating the lab. This
involves working through a pre-made lab to ensure they can help students
apply concepts discussed in lecture to what's being assigned. In
addition, both must hold two hours of office hours each week and have
grading responsibilities assigned throughout the semester.

Head TA responsibilities can generally be categorized as one of the
following: Administrative or Pedagogical. Administrative
responsibilities include the organization and distribution of TA
responsibilities throughout the semester. It is imperative that the head
TA and instructor clearly communicate expectations with each other to
establish exactly how rules and responsibilities that are given to the
TAs are assigned. Administrative duties include reminding other TAs
about bi-weekly payroll deadlines and ensure TAs are working their
alloted hours per week (and not more). Within these alloted hours, a
head TA distributes grading assignments and deadlines to both lab
helpers and leaders per week. They make sure all TAs complete grading
within a week and spot check the grading accuracy and quality of all
written feedback given. (insert 1-2 sentences about experience).
Pedagogically, head TAs are responsible for creating or reviewing answer
keys and grading rubrics for homework and lab assessments as the
instructor sees fit. Each head TA is also assigned to instruct one lab
section during the semester. Before becoming a Head TA, there are
additional

The course organizer is expected to work across each section of STA199,
instead of working with a single instructor. Their responsibilities
include creating rubrics for and working through homework and lab
assignments. Additionally the course organizer, along with the
instructor, answers real time questions virtually during labs from lab
leaders. Questions often range from content related to technical
questions about GitHub and R. Finally, the course organizer is
responsible for handling all requested assignment extension requests
from students. This includes filing away student exemptions, providing
extensions for extreme circumstances, and enforcing the late work policy
outlined in the syllabus when necessary.

(Paragraph on flexibility in team structure)

Through my experience as an instructor working with this designed team,
it is imperative that everyone is communicating with each other. A team
with many different roles poses risk for the instructor to be unaware of
how or what decisions are being enacted at the grading and lab levels of
the course. Thus, it is recommended that the instructor trains everyone
on the teaching team to use a communication system that allows every
member to communicate any questions they may have, or decisions they
make to the entire team. In the past, we have used the software
\emph{Slack}, with appropriately named channels such as
\emph{grading-questions} where TAs can post examples and questions about
grading and the instructor can clearly state their expectations.
Further, it should be noted that the head TA should not be treated as a
``bridge of communication'' for the instructor to the rest of the
teaching team. It is critical that that the instructor is in consistent
contact with all members of the teaching team in making sure all lab
leaders and helpers understand the course content, upcoming assignments,
and know what's expected of them in their assigned role. We recommend
holding a weekly meeting with all members of the teaching team to ensure
this. When members are unable to come, it is an expectation that they
watch a recorded video of the meeting and reach out if they have any
questions about what was discussed.

\hypertarget{sec-tech}{%
\section{Technology}\label{sec-tech}}

In this section we will detail the computing infrastructure used in
STA199 used to create and a course such as STA199. This includes details
on how to use R, R-studio, and GitHub's from the instructor's
perspective to set up lectures, AEs, labs and homework. First, we start
with the student information necessary to collect and steps that must be
taken before creation can take place.

\hypertarget{setting-up-a-github-account}{%
\subsection{Setting up a GitHub
account}\label{setting-up-a-github-account}}

We can use R, R-studio, and GitHub to create interactive lessons, and
assign pre-created assessments to individual or groups of students. To
do so, we must instruct students to create a GitHub account. This is
done of the first day of class, and often, students are given time
during class to sign up. Following tips from ``Happy Git with R''
(cite), we suggest students do the following when creating their name:

-- Incorporate your actual name

-- Reuse your username from other contexts if you can

-- Pick a username you will be comfortable revealing to your future boss

-- Be as unique as possible in as few characters as possible. Shorter is
better than longer

-- Avoid words with special meaning in programming (i.e., NA)

Once students create their account, we suggest getting this information
from them in a survey. This normally can be done through your learning
management system. It is critical to reiterate to students that spelling
and capitalization matter when answer this question. We suggest asking
the question as follows:

\begin{figure}

{\centering \includegraphics{images/github.question.png}

}

\end{figure}

If you, as the instructor, do not have a GitHub account, you will need
to create one as well. This student information will be needed to create
your GitHub organization for your course.

\hypertarget{github-organization}{%
\subsection{GitHub organization}\label{github-organization}}

GitHub organizations are shared accounts where instructors and students
can collaborate across many projects at once. Using your account, you
can create a new GitHub organization by clicking on your profile icon in
the upper right hand corner, go to \emph{Settings}, \emph{Access},
\emph{New Organization}. It's suggested to name this organization the
name of your class and the current semester you teaching in (i.e.,
sta199-s23). One of the many benefits of teaching through GitHub is the
ability to re-use what you currently create as a template for subsequent
semesters. Once your organization is created we can use packages within
R and R-studio to efficiently invite students and create assessments for
them.

\hypertarget{setting-up-r-r-studio}{%
\subsection{Setting Up R \& R-Studio}\label{setting-up-r-r-studio}}

R is a statistical programming language for computing, modeling, and
data visualization, while R-studio is an integrated development
environment for R (cite). Both are freely available to download and use.
In an introductory course, it is recommended to minimize student
frustration and distraction through the use of pre-packaged
computational infrastructure (Çetinkaya-Rundel and Rundel, 2018). Per
this recommendation, STA199 has students use R and R-studio through a
\emph{Duke Container}. Duke containers provides instructors at Duke
university the opportunity to facilitate the use of different software,
such as R and R-studio, through an online container instead of needing
students to locally download both programs. Additionally, instructors
have the ability to manage and install packages students will need for
the semester, helping provide a neat and well organized starting
experience with a new statistical language. (insert discussion on how
this is done).

(Thoughts: Transition to R-studio cloud discussion? What alternatives
can instructors use to imitate Duke containers if this is something they
do not have access to at their university.)

\hypertarget{using-r-r-studio-to-manage-github-organization}{%
\subsection{Using R \& R-Studio to manage GitHub
organization}\label{using-r-r-studio-to-manage-github-organization}}

(insert discussion / how-to on ghclass): includes invitations.

(set the stage for writing on how to create repos for students is
subsequent sections \emph{which will be a good segway})

. . .

R, R-studio, and GitHub aids in the creation and implementation our data
science pedagogy. This includes the creation and distribution of
in-class AEs, lectures, labs, and assessments. In the following sections
we describe in detail both how to create, structure, and distribute
materials used in STA199.

\begin{figure}

{\centering \includegraphics{images/pedagogy.png}

}

\end{figure}

\hypertarget{sec-ped}{%
\section{Pedagogy}\label{sec-ped}}

-- Reader should want to continue reading by thinking ``so how do I
teach this''

In this course, we have chosen a combination of teaching methods,
interactive activities, and learning assessments to help best prepare
introductory data science students the tools they need to be successful
outside of university or in future coursework. Our pedagogy includes
facilitating in-class AEs, facilitating lectures, running a lab, and
assigning assessments to provide students an opportunity to show what
they've learned.

\hypertarget{application-exercies-aes}{%
\subsection{Application Exercies (AEs)}\label{application-exercies-aes}}

\hypertarget{lecture}{%
\subsection{Lecture}\label{lecture}}

\hypertarget{labs}{%
\subsection{Labs}\label{labs}}

\hypertarget{assessments}{%
\subsection{Assessments}\label{assessments}}


\renewcommand\refname{Discussion}
  \bibliography{bibliography.bib}


\end{document}
