% Options for packages loaded elsewhere
\PassOptionsToPackage{unicode}{hyperref}
\PassOptionsToPackage{hyphens}{url}
\PassOptionsToPackage{dvipsnames,svgnames,x11names}{xcolor}
%
\documentclass[
  12pt]{article}

\usepackage{amsmath,amssymb}
\usepackage{lmodern}
\usepackage{iftex}
\ifPDFTeX
  \usepackage[T1]{fontenc}
  \usepackage[utf8]{inputenc}
  \usepackage{textcomp} % provide euro and other symbols
\else % if luatex or xetex
  \usepackage{unicode-math}
  \defaultfontfeatures{Scale=MatchLowercase}
  \defaultfontfeatures[\rmfamily]{Ligatures=TeX,Scale=1}
\fi
% Use upquote if available, for straight quotes in verbatim environments
\IfFileExists{upquote.sty}{\usepackage{upquote}}{}
\IfFileExists{microtype.sty}{% use microtype if available
  \usepackage[]{microtype}
  \UseMicrotypeSet[protrusion]{basicmath} % disable protrusion for tt fonts
}{}
\makeatletter
\@ifundefined{KOMAClassName}{% if non-KOMA class
  \IfFileExists{parskip.sty}{%
    \usepackage{parskip}
  }{% else
    \setlength{\parindent}{0pt}
    \setlength{\parskip}{6pt plus 2pt minus 1pt}}
}{% if KOMA class
  \KOMAoptions{parskip=half}}
\makeatother
\usepackage{xcolor}
\setlength{\emergencystretch}{3em} % prevent overfull lines
\setcounter{secnumdepth}{5}
% Make \paragraph and \subparagraph free-standing
\ifx\paragraph\undefined\else
  \let\oldparagraph\paragraph
  \renewcommand{\paragraph}[1]{\oldparagraph{#1}\mbox{}}
\fi
\ifx\subparagraph\undefined\else
  \let\oldsubparagraph\subparagraph
  \renewcommand{\subparagraph}[1]{\oldsubparagraph{#1}\mbox{}}
\fi


\providecommand{\tightlist}{%
  \setlength{\itemsep}{0pt}\setlength{\parskip}{0pt}}\usepackage{longtable,booktabs,array}
\usepackage{calc} % for calculating minipage widths
% Correct order of tables after \paragraph or \subparagraph
\usepackage{etoolbox}
\makeatletter
\patchcmd\longtable{\par}{\if@noskipsec\mbox{}\fi\par}{}{}
\makeatother
% Allow footnotes in longtable head/foot
\IfFileExists{footnotehyper.sty}{\usepackage{footnotehyper}}{\usepackage{footnote}}
\makesavenoteenv{longtable}
\usepackage{graphicx}
\makeatletter
\def\maxwidth{\ifdim\Gin@nat@width>\linewidth\linewidth\else\Gin@nat@width\fi}
\def\maxheight{\ifdim\Gin@nat@height>\textheight\textheight\else\Gin@nat@height\fi}
\makeatother
% Scale images if necessary, so that they will not overflow the page
% margins by default, and it is still possible to overwrite the defaults
% using explicit options in \includegraphics[width, height, ...]{}
\setkeys{Gin}{width=\maxwidth,height=\maxheight,keepaspectratio}
% Set default figure placement to htbp
\makeatletter
\def\fps@figure{htbp}
\makeatother

\addtolength{\oddsidemargin}{-.5in}%
\addtolength{\evensidemargin}{-1in}%
\addtolength{\textwidth}{1in}%
\addtolength{\textheight}{1.7in}%
\addtolength{\topmargin}{-1in}%
\makeatletter
\makeatother
\makeatletter
\makeatother
\makeatletter
\@ifpackageloaded{caption}{}{\usepackage{caption}}
\AtBeginDocument{%
\ifdefined\contentsname
  \renewcommand*\contentsname{Table of contents}
\else
  \newcommand\contentsname{Table of contents}
\fi
\ifdefined\listfigurename
  \renewcommand*\listfigurename{List of Figures}
\else
  \newcommand\listfigurename{List of Figures}
\fi
\ifdefined\listtablename
  \renewcommand*\listtablename{List of Tables}
\else
  \newcommand\listtablename{List of Tables}
\fi
\ifdefined\figurename
  \renewcommand*\figurename{Figure}
\else
  \newcommand\figurename{Figure}
\fi
\ifdefined\tablename
  \renewcommand*\tablename{Table}
\else
  \newcommand\tablename{Table}
\fi
}
\@ifpackageloaded{float}{}{\usepackage{float}}
\floatstyle{ruled}
\@ifundefined{c@chapter}{\newfloat{codelisting}{h}{lop}}{\newfloat{codelisting}{h}{lop}[chapter]}
\floatname{codelisting}{Listing}
\newcommand*\listoflistings{\listof{codelisting}{List of Listings}}
\makeatother
\makeatletter
\@ifpackageloaded{caption}{}{\usepackage{caption}}
\@ifpackageloaded{subcaption}{}{\usepackage{subcaption}}
\makeatother
\makeatletter
\@ifpackageloaded{tcolorbox}{}{\usepackage[many]{tcolorbox}}
\makeatother
\makeatletter
\@ifundefined{shadecolor}{\definecolor{shadecolor}{rgb}{.97, .97, .97}}
\makeatother
\makeatletter
\makeatother
\ifLuaTeX
  \usepackage{selnolig}  % disable illegal ligatures
\fi
\usepackage[]{natbib}
\bibliographystyle{agsm}
\IfFileExists{bookmark.sty}{\usepackage{bookmark}}{\usepackage{hyperref}}
\IfFileExists{xurl.sty}{\usepackage{xurl}}{} % add URL line breaks if available
\urlstyle{same} % disable monospaced font for URLs
\hypersetup{
  pdftitle={Introductory Data Science: A Blueprint to Navigate Curricular, Pedagogical, and Computational Challenges},
  pdfauthor={Elijah Meyer; Mine Çetinkaya-Rundel},
  pdfkeywords={Data Science, Curriculum, keyword 3},
  colorlinks=true,
  linkcolor={blue},
  filecolor={Maroon},
  citecolor={Blue},
  urlcolor={Blue},
  pdfcreator={LaTeX via pandoc}}


\begin{document}


\def\spacingset#1{\renewcommand{\baselinestretch}%
{#1}\small\normalsize} \spacingset{1}


%%%%%%%%%%%%%%%%%%%%%%%%%%%%%%%%%%%%%%%%%%%%%%%%%%%%%%%%%%%%%%%%%%%%%%%%%%%%%%

\date{January 17, 2023}
\title{\bf Introductory Data Science: A Blueprint to Navigate
Curricular, Pedagogical, and Computational Challenges}
\author{
Elijah Meyer\\
Department of Statistics, Duke University\\
and\\Mine Çetinkaya-Rundel\\
Department of Statistics, Duke University\\
}
\maketitle

\bigskip
\bigskip
\begin{abstract}
The text of your abstract. 200 or fewer words.
\end{abstract}

\noindent%
{\it Keywords:} Data Science, Curriculum, keyword 3
\vfill

\newpage
\spacingset{1.9} % DON'T change the spacing!
\ifdefined\Shaded\renewenvironment{Shaded}{\begin{tcolorbox}[interior hidden, sharp corners, frame hidden, borderline west={3pt}{0pt}{shadecolor}, enhanced, boxrule=0pt, breakable]}{\end{tcolorbox}}\fi

\hypertarget{sec-intro}{%
\section{Introduction}\label{sec-intro}}

The need for training in data science is clear. An estimated 11.5
million new data science jobs are projected to be created by 2026, while
employment of data scientists is projected to grow by 36 percent from
2021 to 2031 (U.S. Bureau of Labor Statistics, 2023). In response,
college and university faculty are starting to develop and implement
data science courses (Schwab-McCoy, Baker, \& Gasper, 2020). The purpose
of this paper is to continue to promote conversations and provide
recommendations around the development and implementation of data
science courses through the lens of a modernized introductory data
science course taught at Duke University.

This course is designed for students with little to no statistics, data
science, or coding experience, a common hurdle identified by faculty
when trying to adopt data science (Schwab-McCoy, Baker, \& Gasper,
2020). By the end of this course, students are expected and able to
clean, investigate, and communicate with data in a reproduceable manner
while answering a targeted research question. Detailed learning
objectives of this course include learning to explore, visualize, and
analyze data in a reproducible and shareable manner through the use of
R-studio and GitHub (cite). Through these programs, students gain
experience in data wrangling and munging, exploratory data analysis,
predictive modeling, and data visualization. These experiences are
generated through a Prepare, Practice, and Perform learning model
emersed in real-world questions and data.

To continue and progress the conversation around data science courses,
we first examine current curriculum recommendations for undergraduate
programs in data science, discuss the current state in undergraduate
introductory data science curricula, and review current pedagogical
recommendations on how to teach such courses. Next, we discuss the
creation and implementation of curricular and pedagogical decisions made
in designing the introductory data science course at Duke University.
This includes detailing the Prepare, Practice, and Perform format, to
support a large class of students with a diverse background in
statistics, data science, and coding experience. Within this format, we
provide examples of and describe activities and assessments given both
in and outside of class. We extend discussions and provide
recommendations for implementing and integrating computing tools, such
as R-studio and GitHub, in a data science course through our experiences
in our course. Lastly, we discuss challenges, and provide insight to
help faculty wanting to adopt or adapt a course similar to introductory
to data science at Duke University.

\hypertarget{data-science-a-review}{%
\section{Data Science: A Review}\label{data-science-a-review}}

Although the definition of data science is fluid, it can be generally
defined the process transforing raw data into understanding, insight,
and knowledge (Wickham \& Grolemund). In practice, data scientists often
describe their practice as a means to ``gain insights'' or ``extract
meaning'' from data (Hernán \ldots{} , 2019). In the following sections,
we describe the current curricular and pedagogical recommendations for
designing a course in data science.

\hypertarget{curriculum-what-is-being-taught}{%
\subsection{Curriculum (what is being
taught)}\label{curriculum-what-is-being-taught}}

Until recently, data modeling made up the majority of data science
curricula through classes housed in statistics and mathematics
departments (Donoho, 2017). However, there has been a call to create
more well-rounded curricula that better align students of with our
working definition of data science. In 2017, Curriculum Guidelines for
Undergraduate Programs in Data Science provided six major
recommendations as to what practitioners of data science should be
competent in: Computational and statistical thinking; Mathematical
foundations; Model building and assessment; Algorithms and software
foundation; Data curation; Knowledge transference---communication and
responsibility (Veaux, et al., 2017). These guiding pillars offer an
approach for students to develop into problem solvers who interact with,
investigate, and make meaning with data. Other visions of data science
include more broad divisions of activity: The activities of GDS are
classified into six divisions: Data Gathering, Preparation, and
Exploration; Data Representation and Transformation; Computing with
Data; Data Modeling; Data Visualization and Presentation; and Science
about Data Science (Donoh, 2017).

The Association for Computing Machinery (ACM) Education Council is a
task force to explore and expand discipline-specific conversations
around the field of data science (cite). This council acknowledges that
data science curricula can be flexible, but should emphasize suggests
that data science curricula should include applications designed towards
building skills in computing, statistics, machine learning and
mathematics.

Among existing curricular recommendations, there are computational and
technological recommendations for data science classrooms. This includes
the use of GitHub to ensure the concept of reproducibility and
incorporating quantitative programming environments (such as R) (Donoho,
2017; Beckman et al., 2021). Among all recommendations, it is critical
to immerse students in real-world data and provide open-ended projects.
This allows students ``early exposure to and experience with the full
data science cycle'', in a practical real-world context (Lui \& Huang,
2017; CETINKAYA-RUNDEL, et al., 2022, pg. 3)

\hypertarget{pedagogical-how-to-teach}{%
\subsection{Pedagogical (how to teach)}\label{pedagogical-how-to-teach}}

Research has shown that the adoption, integration, and implementation of
active learning teaching techniques can help promote student learning,
achievement, and confidence in mathematics (e.g., Freeman et al., 2004).


  \bibliography{bibliography.bib}


\end{document}
