% Options for packages loaded elsewhere
\PassOptionsToPackage{unicode}{hyperref}
\PassOptionsToPackage{hyphens}{url}
\PassOptionsToPackage{dvipsnames,svgnames,x11names}{xcolor}
%
\documentclass[
  12pt]{article}

\usepackage{amsmath,amssymb}
\usepackage{lmodern}
\usepackage{iftex}
\ifPDFTeX
  \usepackage[T1]{fontenc}
  \usepackage[utf8]{inputenc}
  \usepackage{textcomp} % provide euro and other symbols
\else % if luatex or xetex
  \usepackage{unicode-math}
  \defaultfontfeatures{Scale=MatchLowercase}
  \defaultfontfeatures[\rmfamily]{Ligatures=TeX,Scale=1}
\fi
% Use upquote if available, for straight quotes in verbatim environments
\IfFileExists{upquote.sty}{\usepackage{upquote}}{}
\IfFileExists{microtype.sty}{% use microtype if available
  \usepackage[]{microtype}
  \UseMicrotypeSet[protrusion]{basicmath} % disable protrusion for tt fonts
}{}
\makeatletter
\@ifundefined{KOMAClassName}{% if non-KOMA class
  \IfFileExists{parskip.sty}{%
    \usepackage{parskip}
  }{% else
    \setlength{\parindent}{0pt}
    \setlength{\parskip}{6pt plus 2pt minus 1pt}}
}{% if KOMA class
  \KOMAoptions{parskip=half}}
\makeatother
\usepackage{xcolor}
\setlength{\emergencystretch}{3em} % prevent overfull lines
\setcounter{secnumdepth}{5}
% Make \paragraph and \subparagraph free-standing
\ifx\paragraph\undefined\else
  \let\oldparagraph\paragraph
  \renewcommand{\paragraph}[1]{\oldparagraph{#1}\mbox{}}
\fi
\ifx\subparagraph\undefined\else
  \let\oldsubparagraph\subparagraph
  \renewcommand{\subparagraph}[1]{\oldsubparagraph{#1}\mbox{}}
\fi


\providecommand{\tightlist}{%
  \setlength{\itemsep}{0pt}\setlength{\parskip}{0pt}}\usepackage{longtable,booktabs,array}
\usepackage{calc} % for calculating minipage widths
% Correct order of tables after \paragraph or \subparagraph
\usepackage{etoolbox}
\makeatletter
\patchcmd\longtable{\par}{\if@noskipsec\mbox{}\fi\par}{}{}
\makeatother
% Allow footnotes in longtable head/foot
\IfFileExists{footnotehyper.sty}{\usepackage{footnotehyper}}{\usepackage{footnote}}
\makesavenoteenv{longtable}
\usepackage{graphicx}
\makeatletter
\def\maxwidth{\ifdim\Gin@nat@width>\linewidth\linewidth\else\Gin@nat@width\fi}
\def\maxheight{\ifdim\Gin@nat@height>\textheight\textheight\else\Gin@nat@height\fi}
\makeatother
% Scale images if necessary, so that they will not overflow the page
% margins by default, and it is still possible to overwrite the defaults
% using explicit options in \includegraphics[width, height, ...]{}
\setkeys{Gin}{width=\maxwidth,height=\maxheight,keepaspectratio}
% Set default figure placement to htbp
\makeatletter
\def\fps@figure{htbp}
\makeatother

\addtolength{\oddsidemargin}{-.5in}%
\addtolength{\evensidemargin}{-1in}%
\addtolength{\textwidth}{1in}%
\addtolength{\textheight}{1.7in}%
\addtolength{\topmargin}{-1in}%
\makeatletter
\makeatother
\makeatletter
\makeatother
\makeatletter
\@ifpackageloaded{caption}{}{\usepackage{caption}}
\AtBeginDocument{%
\ifdefined\contentsname
  \renewcommand*\contentsname{Table of contents}
\else
  \newcommand\contentsname{Table of contents}
\fi
\ifdefined\listfigurename
  \renewcommand*\listfigurename{List of Figures}
\else
  \newcommand\listfigurename{List of Figures}
\fi
\ifdefined\listtablename
  \renewcommand*\listtablename{List of Tables}
\else
  \newcommand\listtablename{List of Tables}
\fi
\ifdefined\figurename
  \renewcommand*\figurename{Figure}
\else
  \newcommand\figurename{Figure}
\fi
\ifdefined\tablename
  \renewcommand*\tablename{Table}
\else
  \newcommand\tablename{Table}
\fi
}
\@ifpackageloaded{float}{}{\usepackage{float}}
\floatstyle{ruled}
\@ifundefined{c@chapter}{\newfloat{codelisting}{h}{lop}}{\newfloat{codelisting}{h}{lop}[chapter]}
\floatname{codelisting}{Listing}
\newcommand*\listoflistings{\listof{codelisting}{List of Listings}}
\makeatother
\makeatletter
\@ifpackageloaded{caption}{}{\usepackage{caption}}
\@ifpackageloaded{subcaption}{}{\usepackage{subcaption}}
\makeatother
\makeatletter
\@ifpackageloaded{tcolorbox}{}{\usepackage[many]{tcolorbox}}
\makeatother
\makeatletter
\@ifundefined{shadecolor}{\definecolor{shadecolor}{rgb}{.97, .97, .97}}
\makeatother
\makeatletter
\makeatother
\ifLuaTeX
  \usepackage{selnolig}  % disable illegal ligatures
\fi
\usepackage[]{natbib}
\bibliographystyle{agsm}
\IfFileExists{bookmark.sty}{\usepackage{bookmark}}{\usepackage{hyperref}}
\IfFileExists{xurl.sty}{\usepackage{xurl}}{} % add URL line breaks if available
\urlstyle{same} % disable monospaced font for URLs
\hypersetup{
  pdftitle={Introductory Data Science: A Blueprint to Navigate Curricular, Pedagogical, and Computational Challenges},
  pdfauthor={Elijah Meyer; Mine Çetinkaya-Rundel},
  pdfkeywords={Data Science, Curriculum, Pedagogy},
  colorlinks=true,
  linkcolor={blue},
  filecolor={Maroon},
  citecolor={Blue},
  urlcolor={Blue},
  pdfcreator={LaTeX via pandoc}}


\begin{document}


\def\spacingset#1{\renewcommand{\baselinestretch}%
{#1}\small\normalsize} \spacingset{1}


%%%%%%%%%%%%%%%%%%%%%%%%%%%%%%%%%%%%%%%%%%%%%%%%%%%%%%%%%%%%%%%%%%%%%%%%%%%%%%

\date{January 18, 2023}
\title{\bf Introductory Data Science: A Blueprint to Navigate
Curricular, Pedagogical, and Computational Challenges}
\author{
Elijah Meyer\\
Department of Statistics, Duke University\\
and\\Mine Çetinkaya-Rundel\\
Department of Statistics, Duke University\\
}
\maketitle

\bigskip
\bigskip
\begin{abstract}
The text of your abstract. 200 or fewer words.
\end{abstract}

\noindent%
{\it Keywords:} Data Science, Curriculum, Pedagogy
\vfill

\newpage
\spacingset{1.9} % DON'T change the spacing!
\ifdefined\Shaded\renewenvironment{Shaded}{\begin{tcolorbox}[enhanced, borderline west={3pt}{0pt}{shadecolor}, sharp corners, boxrule=0pt, breakable, interior hidden, frame hidden]}{\end{tcolorbox}}\fi

\hypertarget{sec-intro}{%
\section{Introduction}\label{sec-intro}}

The increasing volume of enrollment of data science students
\citep{Redmond2022}

\citep{Galyardt14mmm}

requires that statistics and data science educators commit to developing
modern curriculum in order to help students be successful. This rapid
growth is largely motivated by industries evolving towards data-driven
decision making, requiring that data science graduate possess the tools
required to make those kinds of decisions. An estimated 11.5 million new
data science jobs are projected to be created by 2026, while employment
of data scientists is projected to grow by 36 percent from 2021 to 2031
(U.S. Bureau of Labor Statistics, 2023). Despite the demand, colleges
are still struggling with what a modern data science curriculum should
look like (Schwab-McCoy, Baker \& Gasper, 2020). To this point, much
more thought, work, and discussions need to take place there before a
consensus is reached on what should a modern data science curricula, to
best prepare students, should be. In this paper, we add to the
discussion by describing a modernized data science curricula for an
introductory data science course at Duke University.

This course is designed for students with little to no statistics, data
science, or coding experience, a common hurdle identified by faculty
when trying to implement a data science course (Schwab-McCoy, Baker, \&
Gasper, 2020). By the end of this course, students are expected and able
to clean, investigate, and communicate with data in a reproducible
manner while answering a targeted research question. Detailed learning
objectives of this course include learning to explore, visualize, and
analyze data in a reproducible and shareable manner through the use of
R-studio and GitHub (R-Core Team, 2022; GitHub, 2022). Through these
programs, students gain experience in data wrangling, exploratory data
analysis, predictive modeling, and data visualization. These experiences
are generated through the \emph{Kaplan Way}. The Kaplan Way is a
learning model ``that combines a scientific, evidence-based design
philosophy with a straightforward educational approach to learning''
(https://www.schweser.com/about-kaplan/philosophy). This learning model
posits a three-phase learning strategy: Prepare, Practice, and Perform.
During each of these phases, students are equipped with the appropriate
tools to acquire knowledge, be given an opportunity to apply what they
know, and to demonstrate mastery of the tasks at hand.

In this paper, we examine current curriculum recommendations for
undergraduate programs in data science and review current pedagogical
recommendations on how to teach such courses. Next, we discuss the
creation and implementation of curricular and pedagogical decisions made
in designing the introductory data science course at Duke University.
This includes detailing the implementation of the Kaplan Way learning
model, to support a large class of students with a diverse background in
statistics, data science, and coding experience. Within this format, we
provide examples of and describe activities and assessments given both
in and outside of class. We extend discussions and provide
recommendations for implementing and integrating computing tools, such
as R-studio and GitHub, through our experiences in our course. Lastly,
we discuss challenges, and provide insight to help faculty wanting to
adopt or adapt a course similar to introductory to data science at Duke
University. The purpose of this paper is to continue the discussion, and
present a modernized curricula for an introductory data science course
at Duke University, and the pedagogical decisions to help best equip
students with the data science skills necessary for future classes.

\hypertarget{sec-review}{%
\section{Data Science: A Review}\label{sec-review}}

Although the definition of data science is fluid, it can be generally
defined the process transforming raw data into understanding, insight,
and knowledge (Wickham \& Grolemund, 2022). In practice, data scientists
often describe their work as a means to ``gain insights'' or ``extract
meaning'' from data (Hernán \ldots{} , 2019). In the following sections,
we describe the current curricular and pedagogical recommendations for
designing a course in data science.

\hypertarget{sec-curr}{%
\subsection{Curriculum}\label{sec-curr}}

Until recently, data modeling made up the majority of data science
curricula through classes housed in statistics and mathematics
departments (Donoho, 2017). However, there has been a call to create
more well-rounded curricula that better unifies methodology in
statistics, mathematics, computer science, and machine learning to
prepare students to understand, analyze, and communicate with data. In
2017, Curriculum Guidelines for Undergraduate Programs in Data Science
provided six major recommendations as to what practitioners of data
science should be competent in: Computational and statistical thinking;
Mathematical foundations; Model building and assessment; Algorithms and
software foundation; Data curation; Knowledge
transference---communication and responsibility (Veaux, et al., 2017).
These guiding pillars offer an approach for students to develop into
problem solvers who interact with, investigate, and make meaning with
data. Other visions of data science include more broad divisions of
class activity. This includes \emph{greater data science}. Coined by
Chambers and Clevland, the activities of \emph{greater data science} are
classified into six divisions: Data Gathering, Preparation, and
Exploration; Data Representation and Transformation; Computing with
Data; Data Modeling; Data Visualization and Presentation; and Science
about Data Science (Donoh, 2017). They suggest that these activities can
be used to guide and assess if data science programs are adequately
addressing the entirety of the field.

Additionally, a task force, titled the Association for Computing
Machinery (ACM) Education Council, explores and expands
discipline-specific conversations around the field of data science
(Danyluk \& Leidig, 2021). This council acknowledges that data science
curricula can be flexible, but suggests that data science curricula
should include applications designed towards building skills in
computing, statistics, machine learning and mathematics.

Among existing curricular recommendations, there are computational and
technological recommendations for data science classrooms. This includes
the use of GitHub to ensure the concept of reproducibility and
incorporating quantitative programming environments (such as R), to
better work with data (Donoho, 2017; Beckman et al., 2021). Among all
recommendations, it is suggested to immerse students in real-world data
and provide open-ended projects. This allows students ``early exposure
to and experience with the full data science cycle'', in a practical
real-world context (Lui \& Huang, 2017; CETINKAYA-RUNDEL, et al., 2022,
pg. 3).

\hypertarget{sec-ped}{%
\subsection{Pedagogical}\label{sec-ped}}

Although not specific to data science, national guidelines have
recommended the use of more student-centered teaching techniques, such
as active learning, in statistics and mathematics classrooms, to better
engage students in the learning process (Carver et al., 2016; Conference
Board of Mathematical Sciences, 2016). The Guidelines for Assessment and
Instruction in Statistics Education College Report further stresses the
importance of active learning in the mathematical sciences, encouraging
instructors to use active learning to foster and enhance students'
understanding and communication of topics taught (Carver et al., 2016).
In general, research has shown that the adoption, integration, and
implementation of active learning teaching techniques can help promote
student learning, achievement, and confidence in science courses (e.g.,
Freeman et al., 2014). Teaching data science courses through the use of
active learning can help better facilitate a more complete understanding
of working with data.

As the volume of data continues to grow, the need for those who can make
meaning of data are clear. Thus, it is critical that data science
education provides programs that adequately prepare and train students
in the field of data science. This article adds to the existing
literature by describing a modernized introductory data science course
and how its taught, at Duke University.


  \bibliography{bibliography.bib}


\end{document}
